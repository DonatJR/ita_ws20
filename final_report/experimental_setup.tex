\subsection{Experimental Setup}
\subcomment{Written by Jonas Reinwald}
Before implementing our final pipeline we used Jupyter notebooks to experiment with various clustering algorithms, to get a feel for their behavior and to find a set of good performing parameters.

The insights and written code from these experiments were then used to create the clustering pipeline. It performs the following steps in succession of each other:
\begin{enumerate}
    \item Loading the data
    \item Setting up the preprocessor and applying it to the data
    \item Perform clustering on the preprocessed data
    \item Extracting and saving results from the generated clusters
    \item Extracting top words for each cluster for automatic labelling
    \item (Optional) Evaluating the results by comparing them with ground truth data and calculating evaluation metrics
\end{enumerate}

The code must be provided with a configuration file which includes options to alter the pipeline execution. These options include but are not limited to the data input and output path, the library and type of preprocessing which is to be done on the input data, the kind of (optional) dimensionality reduction and the clustering method, with some changeable parameters, to use.
We provide one default configuration file for all included clustering approaches with which the results can be reproduced. 