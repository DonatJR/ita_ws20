\subsection{Introduction}

\subcomment{Written by Daniela Fichiu}
If you type a query like "how do I open the terminal on mac?" into Google, you already know what you want to find: a set of steps that will open a terminal on the screen.

Most would say doing homework is, on the best days, an unpleasant affair. But everyone would agree that writing an assignment for that one course you've always skipped is a chore. You most certainly do not understand the airy slides. Your only option is to search for materials on your own - extensive research is needed to find the materials that can fill all your knowledge gaps.

A student who wants to know more about text analytics might search for "text analytics" on Google Scholar. The query yields back 1.650.000 results, the top three being "Text Analytics with Python," "Text analytics in social media," and "Semantic interaction for visual text analytics." A subsequent query like "text analytics topics" yields back results with the following titles: "Analyzing educational comments for topics and sentiments: A text analytics approach" or "A text analytics approach for online retailing service improvement: Evidence from Twitter."

We know how time-consuming it is to spend hours on search engines or websites like Research Gate or Google Scholar looking for research papers, hoping for the best, but never quite finding the perfect materials.

Our project emerged from the need to find an easy way of exhaustively searching for scholarly literature while bringing to light the relations between the subfields of the research field of interest and other areas. Our goal is to provide a deeper understanding of the material to be searched for and easing up the process of finding information.

We propose a solution that clusters scientific texts into relevant subgroups. The subgroups can then be more easily presented to and explored by people looking for specific topics and terms.

We focus on a subset of around two thousand papers from the field of machine learning. We cluster the research papers, extract a ground truth using the papers' keywords, and compare the clustering results against the ground truth. We also prove that an unbalanced data set can have a significant impact on the clustering results. 

Even with the conclusions written down, we do not see our work as finished. The proposed pipeline could be usable on a broader range of fields. We also intend to balance our data set by adding research papers from other areas.

We also believe that our solution, supported by good group visualization tools and a user-friendly search interface, can be successfully integrated into a metasearch site.