% This file was created with JabRef 2.10.
% Encoding: ISO8859_1


@InProceedings{An_Approach_to_Clustering_Abstracts,
  Title                    = {An Approach to Clustering Abstracts},
  Author                   = {Alexandrov, Mikhail
and Gelbukh, Alexander
and Rosso, Paolo},
  Booktitle                = {Natural Language Processing and Information Systems},
  Year                     = {2005},

  Address                  = {Berlin, Heidelberg},
  Editor                   = {Montoyo, Andr{\'e}s
and Mu{\'{n}}oz, Rafael
and M{\'e}tais, Elisabeth},
  Pages                    = {275--285},
  Publisher                = {Springer Berlin Heidelberg},

  Abstract                 = {Free access to full-text scientific papers in major digital libraries and other web repositories is limited to only their abstracts consisting of no more than several dozens of words. Current keyword-based techniques allow for clustering such type of short texts only when the data set is multi-category, e.g., some documents are devoted to sport, others to medicine, others to politics, etc. However, they fail on narrow domain-oriented libraries, e.g., those containing all documents only on physics, or all on geology, or all on computational linguistics, etc. Nevertheless, just such data sets are the most frequent and most interesting ones. We propose simple procedure to cluster abstracts, which consists in grouping keywords and using more adequate document similarity measure. We use Stein's MajorClust method for clustering both keywords and documents. We illustrate our approach on the texts from the Proceedings of a narrow-topic conference. Limitations of our approach are also discussed. Our preliminary experiments show that abstracts cannot be clustered with the same quality as full texts, though the achieved quality is adequate for many applications; accordingly, we suggest Makagonov's proposal that digital libraries should provide document images of full texts of the papers (and not only abstracts) for open access via Internet, in order to help in search, classification, clustering, selection, and proper referencing of the papers.},
  ISBN                     = {978-3-540-32110-1}
}

@Article{Document_clustering_of_scientific_texts_using_citation_contexts,
  Title                    = {Document clustering of scientific texts using citation contexts},
  Author                   = {Aljaber, Bader and Stokes, Nicola and Bailey, James and Pei, Jian},
  Journal                  = {Inf. Retr.},
  Year                     = {2010},

  Month                    = {04},
  Pages                    = {101-131},
  Volume                   = {13},

  Doi                      = {10.1007/s10791-009-9108-x}
}

@Article{Clustering_of_scientific_journals,
  Title                    = {Clustering of scientific journals},
  Author                   = {Carpenter, Mark P. and Narin, Francis},
  Journal                  = {Journal of the American Society for Information Science},
  Year                     = {1973},
  Number                   = {6},
  Pages                    = {425-436},
  Volume                   = {24},

  Abstract                 = {Abstract A cluster analysis procedure is described in which 288 journals in the disciplines of physics, chemistry and molecular biology are grouped into clusters. Most of the clusters are easily identified as subdisciplinary subject areas. The data source was the cross citing amongst the journals derived from the Journal Citation Index (JCI), a file derived in turn from the Science Citation Index (SCI)®. The JCI consists of journal by journal tabulation of citings to and from each journal processed in the SCI. Two-step citation maps linking the clusters are presented for each discipline. Within the disciplines the clusters of journals form fully transitive hierarchies with very few relational conflicts.},
  Doi                      = {https://doi.org/10.1002/asi.4630240604},
  Eprint                   = {https://asistdl.onlinelibrary.wiley.com/doi/pdf/10.1002/asi.4630240604},
  Url                      = {https://asistdl.onlinelibrary.wiley.com/doi/abs/10.1002/asi.4630240604}
}

@Article{Rapid_understanding_of_scientific_paper_collections,
  Title                    = {Rapid understanding of scientific paper collections: Integrating statistics, text analytics, and visualization},
  Author                   = {Dunne, Cody and Shneiderman, Ben and Gove, Robert and Klavans, Judith and Dorr, Bonnie},
  Journal                  = {Journal of the American Society for Information Science and Technology},
  Year                     = {2012},
  Number                   = {12},
  Pages                    = {2351-2369},
  Volume                   = {63},

  Abstract                 = {Keeping up with rapidly growing research fields, especially when there are multiple interdisciplinary sources, requires substantial effort for researchers, program managers, or venture capital investors. Current theories and tools are directed at finding a paper or website, not gaining an understanding of the key papers, authors, controversies, and hypotheses. This report presents an effort to integrate statistics, text analytics, and visualization in a multiple coordinated window environment that supports exploration. Our prototype system, Action Science Explorer (ASE), provides an environment for demonstrating principles of coordination and conducting iterative usability tests of them with interested and knowledgeable users. We developed an understanding of the value of reference management, statistics, citation text extraction, natural language summarization for single and multiple documents, filters to interactively select key papers, and network visualization to see citation patterns and identify clusters. A three-phase usability study guided our revisions to ASE and led us to improve the testing methods.},
  Doi                      = {https://doi.org/10.1002/asi.22652},
  Eprint                   = {https://onlinelibrary.wiley.com/doi/pdf/10.1002/asi.22652},
  Keywords                 = {graphs, visualization (electronic), natural language processing},
  Url                      = {https://onlinelibrary.wiley.com/doi/abs/10.1002/asi.22652}
}

@Article{Clustering_scientific_documents_with_topic_modeling,
  Title                    = {Clustering scientific documents with topic modeling},
  Author                   = {Yau, c-k and Porter, Alan and Newman, Nils and Suominen, Arho},
  Journal                  = {Scientometrics},
  Year                     = {2014},

  Month                    = {09},
  Pages                    = {767-786},
  Volume                   = {100},

  Doi                      = {10.1007/s11192-014-1321-8}
}

@Misc{aan,
  Title                    = {ACL Anthology Network (AAN)},

  Keywords                 = {data},
  Owner                    = {hoc2hi},
  Timestamp                = {2020.11.22},
  howpublished                      = {http://aan.how/}
}

@Misc{web_of_science,
  Title                    = {Web of Science},

  Keywords                 = {data},
  Owner                    = {hoc2hi},
  Timestamp                = {2020.11.22},
  howpublished                      = {www.webofknowledge.com}
}

@Misc{cicling,
  Title                    = {CICLing: International Conference on Computational Linguistics and Intelligent Text Processing},

  Keywords                 = {data},
  Owner                    = {hoc2hi},
  Timestamp                = {2020.11.22},
  howpublished                      = {https://www.cicling.org/}
}


@misc{kmeans,
  title = {K-means Clustering: Algorithm, Applications, Evaluation Methods, and Drawbacks},
  author = {Imad Dabbura},
  howpublished = {\url{https://towardsdatascience.com/k-means-clustering-algorithm-applications-evaluation-methods-and-drawbacks-aa03e644b48a}},
  note = {Accessed: 2021-03-08}
}

@MISC{spectral_clustering,
    author = {Ulrike Von Luxburg},
    title = {A Tutorial on Spectral Clustering},
    year = {2007}
}

@article{10.1145/235968.233324,
author = {Zhang, Tian and Ramakrishnan, Raghu and Livny, Miron},
title = {BIRCH: An Efficient Data Clustering Method for Very Large Databases},
year = {1996},
issue_date = {June 1996},
publisher = {Association for Computing Machinery},
address = {New York, NY, USA},
volume = {25},
number = {2},
issn = {0163-5808},
url = {https://doi.org/10.1145/235968.233324},
doi = {10.1145/235968.233324},
abstract = {Finding useful patterns in large datasets has attracted considerable interest recently, and one of the most widely studied problems in this area is the identification of clusters, or densely populated regions, in a multi-dimensional dataset. Prior work does not adequately address the problem of large datasets and minimization of I/O costs.This paper presents a data clustering method named BIRCH (Balanced Iterative Reducing and Clustering using Hierarchies), and demonstrates that it is especially suitable for very large databases. BIRCH incrementally and dynamically clusters incoming multi-dimensional metric data points to try to produce the best quality clustering with the available resources (i.e., available memory and time constraints). BIRCH can typically find a good clustering with a single scan of the data, and improve the quality further with a few additional scans. BIRCH is also the first clustering algorithm proposed in the database area to handle "noise" (data points that are not part of the underlying pattern) effectively.We evaluate BIRCH's time/space efficiency, data input order sensitivity, and clustering quality through several experiments. We also present a performance comparisons of BIRCH versus CLARANS, a clustering method proposed recently for large datasets, and show that BIRCH is consistently superior.},
journal = {SIGMOD Rec.},
month = jun,
pages = {103–114},
numpages = {12}
}

@inproceedings{birch,
author = {Zhang, Tian and Ramakrishnan, Raghu and Livny, Miron},
title = {BIRCH: An Efficient Data Clustering Method for Very Large Databases},
year = {1996},
isbn = {0897917944},
publisher = {Association for Computing Machinery},
address = {New York, NY, USA},
url = {https://doi.org/10.1145/233269.233324},
doi = {10.1145/233269.233324},
booktitle = {Proceedings of the 1996 ACM SIGMOD International Conference on Management of Data},
pages = {103-114},
numpages = {12},
location = {Montreal, Quebec, Canada},
series = {SIGMOD '96}
}

@article{lsa,
author = {Dumais, Susan T.},
title = {Latent semantic analysis},
journal = {Annual Review of Information Science and Technology},
volume = {38},
number = {1},
pages = {188-230},
doi = {https://doi.org/10.1002/aris.1440380105},
url = {https://asistdl.onlinelibrary.wiley.com/doi/abs/10.1002/aris.1440380105},
eprint = {https://asistdl.onlinelibrary.wiley.com/doi/pdf/10.1002/aris.1440380105},
year = {2004}
}

@article{spectral_embedding_paper,
author={M. {Belkin} and P. {Niyogi}},
journal={Neural Computation},
title={Laplacian Eigenmaps for Dimensionality Reduction and Data Representation},
year={2003},
volume={15},
number={6},
pages={1373-1396},
doi={10.1162/089976603321780317}
}

@ARTICLE{affinity_propagation,
    author = {Brendan J. Frey and Delbert Dueck},
    title = {Clustering by passing messages between data points},
    journal = {Science},
    year = {2007},
    volume = {315},
    pages = {2007}
}

@misc{yildirim_2020,
title={Hierarchical Clustering - Explained},
author={Soner Yildirim},
howpublished = {\url{https://towardsdatascience.com/hierarchical-clustering-explained-e58d2f936323}},
note = {Accessed: 2021-03-14}
} 

@inproceedings{dbscan_paper,
author = {Ester, Martin and Kriegel, Hans-Peter and Sander, J\"{o}rg and Xu, Xiaowei},
title = {A Density-Based Algorithm for Discovering Clusters in Large Spatial Databases with Noise},
year = {1996},
publisher = {AAAI Press},
booktitle = {Proceedings of the Second International Conference on Knowledge Discovery and Data Mining},
pages = {226-231},
numpages = {6},
keywords = {arbitrary shape of clusters, efficiency on large spatial databases, handling nlj4-275oise, clustering algorithms},
location = {Portland, Oregon},
series = {KDD'96}
}

@article{optics_paper,
author = {Ankerst, Mihael and Breunig, Markus M. and Kriegel, Hans-Peter and Sander, J\"{o}rg},
title = {OPTICS: Ordering Points to Identify the Clustering Structure},
year = {1999},
issue_date = {June 1999},
publisher = {Association for Computing Machinery},
address = {New York, NY, USA},
volume = {28},
number = {2},
issn = {0163-5808},
url = {https://doi.org/10.1145/304181.304187},
doi = {10.1145/304181.304187},
journal = {SIGMOD Rec.},
month = jun,
pages = {49-60},
numpages = {12},
keywords = {database mining, cluster analysis, visualization}
}

@misc{seif_2021,
title={The 5 Clustering Algorithms Data Scientists Need to Know},
author={George Seif},
howpublished = {\url{https://towardsdatascience.com/the-5-clustering-algorithms-data-scientists-need-to-know-a36d136ef68}},
note = {Accessed: 2021-03-14}
}

@Misc{crossref,
  Title                    = {Crossref},
  Keywords                 = {data},
  Owner                    = {hoc2hi},
  Timestamp                = {2020.11.22},
  howpublished                      = {https://www.crossref.org}
}

@Misc{jmlr,
  Title                    = {The Journal of Machine Learning Research (JMLR)},
  Keywords                 = {data},
  Owner                    = {hoc2hi},
  Timestamp                = {2020.11.22},
  howpublished                      = {https://jmlr.csail.mit.edu}
}

@Misc{p2t,
  author = {Jason Alan Palmer},
  title = {pdftotext},
  year = {2021},
  publisher = {GitHub},
  journal = {GitHub repository},
  howpublished= {hhttps://github.com/jalan/pdftotext}
}

@book{kouzis2016learning,
  title={Learning Scrapy},
  author={Kouzis-Loukas, Dimitrios},
  year={2016},
  publisher={Packt Publishing Ltd}
}

@misc{GROBID,
    title = {GROBID},
    howpublished = {\url{https://github.com/kermitt2/grobid}},
    publisher = {GitHub},
    year = {2008--2021},
    archivePrefix = {swh},
    eprint = {1:dir:dab86b296e3c3216e2241968f0d63b68e8209d3c}
}
