\subsection{Clustering}
\label{clustering_methods}

There are plenty of different clustering algorithms, that perform differently depending on the nature of the data.
The algorithms tested in this project are described below.

\subsubsection{K-Means}
\subsubcomment{Written by Jessica Kaechele}
The goal of K-Means is that data points that are within a cluster are as similar as possible, while two different clusters are as different as possible.
To achieve this, a centroid is randomly selected for each of the $k$ clusters.
Then the sum of squared distance between each data point and the cluster centroid are calculated.
Based on this calculation, the data points are assigned to the closest centroid.
Now, new centroids per cluster are calculated by taking the average of all data points within that cluster.
These steps are repeated until the centroids no longer change.\cite{kmeans}

With K-Means, the number of clusters $k$ must be defined. If $k$ is not known, it can be determined using the elbow method, for example.
In this method, the sum of squared distances between the data points and their cluster centroid is calculated for different $k$.
It can often be seen that the sum of squared distances drops significantly at the beginning and then remains at approximately the same level.
The optimal $k$ can be found at the point where the curve starts to flatten out.\cite{kmeans}

\subsubsection{Spectral Clustering}
\subsubcomment{Written by Jessica Kaechele}
Spectral Clustering already includes a dimensionality reduction.
The first step is to construct a similarity graph.
Each vertice $v_i$ represents a data point $x_i$.
An edge between two vertices $v_i$ and $v_j$ exists if the similarity $s_ij$ is higher than a previously defined threshold.
These similarities can be used as weights $w_ij$ in a weight matrix $W$.
If there is no edge between two vertices, the weight is $0$.
With the help of this weight matrix, a degree matrix can be calculated.
Therefore, the degree is calculated for each vertice by summing the weights:
$$
d_i = \sum_{j=1}^n{w_{ij}}
$$
Then the graph laplacian matrix is calculated by subtracting the weight matrix from the degree matrix:
$$
L = D - W
$$
Afterwards, the first $k$ smallest eigenvectors of the matrix $L$ are calculated.
Finally, the newly created data points are clustered using K-Means.\cite{spectral_clustering}

The graph laplacian matrix can also be calculated as a normalized graph laplacian \cite{spectral_clustering}.
This changes the algorithm slightly, but this is not discussed in detail here, as the functionality of the algorithm does not change.

\subsubsection{BIRCH}
\subsubcomment{Written by Jessica Kaechele}
BIRCH also reduces the scale of the dataset by defining subclusters.
These subclusters are called clustering features(CF).
The information that is retained is the number of data points that are included, the linear sum of the included data points, and the squared sum.
Also, a CF can contain other CFs.
The CFs are now arranged in a tree, with each node containing at most $B$ entries.
An entry in a non-leaf node consists of a CF and a pointer to a child node.
A leaf node consists of at most $L$ CFs.
In addition, there is a threshold $T$, which determines the maximum diameter of the entries of the leaf node.
The diameter is the average pairwise distance within a cluster.

If a new CF or data point is inserted, the tree is descended recursively by selecting the closest child node based on a distance metric.
Once a leaf node is reached, the closest CF entry is searched for, which can absorb the new CF entry without exceeding $T$.
However, if $T$ is exceeded, a new CF entry is appended.
If the leaf node already contains $L$ entries, it must be split.

Now all CF entries of the parent nodes are updated.
In case of a split, a new entry must be added to the parent node.
If $B$ entries are already contained there, this node must also be split again, etc.

When building the tree it is ensured that the complete tree can be loaded in the memory by increasing $T$ if necessary.
The clustering is performed by any clustering algorithm, which clusters the CFs in the leaf nodes.\cite{birch}

\subsubsection{Affinity Propagation}
\subsubcomment{Written by Jonas Reinwald}

\subsubsection{Agglomerative Clustering}
\subsubcomment{Written by Jonas Reinwald}

\subsubsection{DBSCAN}
\subsubcomment{Written by Jonas Reinwald}

\subsubsection{MeanShift}
\subsubcomment{Written by Jonas Reinwald}

\subsubsection{OPTICS}
\subsubcomment{Written by Jonas Reinwald}