\section{Related Work}
\comment{Written by Jessica Kaechele}

There exist various approaches to simplify getting an overview of a research field without extensive research and reading countless papers. By browsing through digital libraries and search engines you can find papers by matching search strings, but to get an overview of an entire research field this is not enough.
% das könnte weg oder kürzer
In \cite{Rapid_understanding_of_scientific_paper_collections} a tool is presented that uses different methods to gain insights into research fields.
Through the citation network, papers can be divided into clusters.
Another possibility to get insights is the citation context because there the key statements of the paper are often summarized concisely.  But it is still necessary to read the whole paper or at least all citation contexts. For many papers, this is still a very large amount of work. To solve this problem Multi-Document Summarization is used. This summarization is only applied to the abstracts and citation contexts.
%

With the help of the mentioned techniques, it should be possible to gain insight into an entire research field more quickly.
In this project, the focus is only on clustering.
For many years, attempts have been made to cluster papers in order to simplify the research. In 1973, for example, it had already been tried to cluster journals by comparing reference patterns and looking at mutual references \cite{Clustering_of_scientific_journals}.

In \cite{Document_clustering_of_scientific_texts_using_citation_contexts} the context of the citations is used in addition to the citations to cluster.
First, a citation has to be recognized and the text has to be extracted on both sides of the citation. Then different clustering approaches are applied and compared. In addition, this technique is also applied to the entire document and compared to the approach of citation context.

In this project, we do also want to use different clustering approaches.
But when examining the citation contexts, access to the entire paper is required. Besides, it is difficult to identify citations in PDFs because of the different ways of citing. Abstracts, on the other hand, are commonly available and no further extraction step is needed.

There are already some approaches that are using abstracts for clustering.
In \cite{Clustering_scientific_documents_with_topic_modeling} abstracts and titles are used.

Two types of pre-processing are performed on the texts. 
Afterward, several topic modelling algorithms are used and the created clusters are named manually.

In our project, a similar pre-processing is done.
Additionally to some topic modelling approaches different clustering algorithms are tested to find out which algorithm fits our dataset best. 
Furthermore, a dimensionality reduction will be tested to improve the clustering and the clusters will be named automatically.

Abstracts are also used in \cite{An_Approach_to_Clustering_Abstracts} to perform clustering, by pre-processing, calculating the cosine similarity, and using different clustering methods.
As dataset 48 abstracts are used, which have been classified by a human.

Again, a similar approach to ours is used. However, with only 48 abstracts, there is a high risk of over-fitting and bias.
This is to be avoided in our project.