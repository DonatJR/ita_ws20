\section{Conclusion}
\comment{Written by Jonas Reinwald}
In our project, we scraped a couple thousand of machine learning papers and created a dataset of unorganized text. We then created clusters with unsupervised algorithms both on the abstracts and the set of keywords.
We used the keyword clusters as ground truth for evaluation. Unfortunately, our results are not as good as we hoped they would be.

In theory, if we could achieve better results, our pipeline could be used to turn unorganized paper collections into more meaningful groups, making them organized and reducing the manual workload of researchers.
If we choose to continue work on this project, a first step towards this goal is to do a conclusive search for the best parameters of each individual algorithm.
Additionally, we have to extend our dataset to a more balanced one and create a better ground-truth from it. This would also allow training of learning-based clustering algorithms which might perform better than our previous approaches.
It is of course vital to then extend the approach to not only work on the subset of machine learning or computer science papers but also on a more general data set for it to be useful to a broader audience.

There are also a number of future applications this project could be used for.
We might be able to use our pipeline to propose keywords when given abstracts, by mapping generated clusters to a set of keywords and then match the given abstract to a cluster-keyword-set.
Furthermore, we think it would be possible to build an API for accessing our results and build visualization tools and website integrations on top of that. 
