\section{Motivation}

Before typing an exact query like "how do I open the terminal on mac?" into a search engine, one already has an exact idea of what he wants to find: a set of steps that will lead to terminal being displayed on the screen. 

When starting a new project or trying to improve existing bodies of work it is often vitally important to do extensive research. This both helps to familiarize oneself with the topic that is to be worked on and provides the necessary new information that is needed to solve the problem at hand.

A student who wants to know more about text analytics might search for "text analytics" on Google Scholar. The query yields back 1.650.000 results, the top three being "Text Analytics with Python", "Text analytics in social media" and "Semantic interaction for visual text analytics". A subsequent query like "text analytics topics" yields back results with the following titles: "Analyzing educational comments for topics and sentiments: A text analytics approach" or "A text analytics approach for online retailing service improvement: Evidence from Twitter". No query has so far returned the expected results.

We know how time-consuming it is to spend hours on search engines or websites like Research Gate or Google Scholar looking for research papers, hoping for the best, but never quite finding what we were looking for.

Our project emerged from the need of finding an easy way of exhaustively searching for scholarly literature, while bringing to light the relations between the subfields of the research field of interest and/or of other fields, thus providing a deeper understanding of the material to be searched for and easing up the process of finding information. We believe that our project will especially benefit people who are just getting started or have little experience with reading or searching for academic papers. 

We want to propose a solution to this by taking scientific texts and clustering them into relevant subgroups, which can then be more easily presented to and explored by people looking for specific topics and terms.

While we intend to first focus on a rather narrow subset of papers from a topic like Deep Learning or something similar, more or less specific subgroups for clustering therein can be explored to find a good balance. The proposed pipeline should later be usable on a broader range of fields.

Furthermore, we think a solution to the stated problem could later be used on a grander scale by building good group visualization tools and providing existing websites with this technology.

In addition, a meta search site incorporating data from these other sources with a common, easily digestible search and presentation interface could be developed.