\section{Motivation}

Motivate your project and state the \textit{real-world problem} you want to solve.

When starting a new project or trying to improve existing bodies of work it is often vitally important to do extensive research.
This both helps to familiarize oneself with the topic that is to be worked on and provides the necessary new information that is needed to solve the problem at hand.

In comparison to more general search engines, where finding information for most topics is rather easy and comfortable, doing research on scientific papers can be more cumbersome.
There are multiple websites with different papers and search interfaces available, and to get a comprehensive overview one has to go through all of them and adapt their `search-procedure' to the available tools.

We want to propose a solution to this by taking scientific texts and clustering them into relevant subgroups, which can then be more easily presented to and explored by people looking for specific topics and terms.
While we intend to first focus on a rather narrow subset of papers from a topic like Deep Learning or something similar, more or less specific subgroups for clustering therein can be explored to find a good balance.
The pipeline should later be usable on a broader range of fields.

Furthermore, we think a solution to the stated problem could later be used on a grander scale by building good group visualization tools and providing existing websites with this technology.
In addition, a meta search site incorporating data from these other sources with a common, easily digestible search and presentation interface could be developed.