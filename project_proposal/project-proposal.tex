% This template was initially provided by Dulip Withanage.
% Modifications for the database systems research group
% were made by Conny Junghans,  Jannik Strötgen and Michael Gertz

\documentclass[
     12pt,         % font size
     a4paper,      % paper format
     BCOR10mm,     % binding correction
     DIV14,        % stripe size for margin calculation
     ]{article}

%%%%%%%%%%%%%%%%%%%%%%%%%%%%%%%%%%%%%%%%%%%%%%%%%%%%%%%%%%%%

% PACKAGES:

% Use German :
\usepackage[english]{babel}
% Input and font encoding
\usepackage[latin1]{inputenc}
\usepackage[T1]{fontenc}
% Index-generation
\usepackage{makeidx}
% Einbinden von URLs:
\usepackage{url}
% Special \LaTex symbols (e.g. \BibTeX):
%\usepackage{doc}
% Include Graphic-files:
\usepackage{graphicx}
% Include doc++ generated tex-files:
%\usepackage{docxx}

% Fuer anderthalbzeiligen Textsatz
\usepackage{setspace}

\usepackage[title]{appendix}

% hyperrefs in the documents
\PassOptionsToPackage{hyphens}{url}\usepackage[bookmarks=true,colorlinks,pdfpagelabels,pdfstartview = FitH,bookmarksopen = true,bookmarksnumbered = true,linkcolor = black,plainpages = false,hypertexnames = false,citecolor = black,urlcolor=black]{hyperref}
%\usepackage{hyperref}

%%%%%%%%%%%%%%%%%%%%%%%%%%%%%%%%%%%%%%%%%%%%%%%%%%%%%%%%%%%%

% OTHER SETTINGS:

% Choose language
\newcommand{\setlang}[1]{\selectlanguage{#1}\nonfrenchspacing}


\begin{document}

% TITLE:
\pagenumbering{roman} 
\begin{titlepage}


\vspace*{1cm}
\begin{center}
\vspace*{3cm}
\textbf{ 
\Large Heidelberg University\\
\smallskip
\Large Institute of Computer Science\\
\smallskip
\Large Database Systems Research Group\\
\smallskip
}

\vspace{3cm}

\textbf{\large Project Proposal for the lecture Text Analytics}

\vspace{0.5\baselineskip}
{\huge
\textbf{Clustering of scientific papers for easy information retrieval}
}
\end{center}

\vfill 

{\large
\begin{tabular}[l]{ll}
Team Member: & Daniela Fichiu, 3552717, BSc Applied Computer Science,\\
  & BSc Mathematics\\
  & daniela.fichiu@stud.uni-heidelberg.de\\
Team Member: & Christian Homeyer, 3606476, PhD Computer Science \\
  & ox182@uni-heidelberg.de\\
Team Member: & Jessica Kaechele, 3588787, MSc Applied Computer Science\\
  & Uo251@stud.uni-heidelberg.de\\
Team Member: & Jonas Reinwald, 3600238, MSc Applied Computer Science\\
  & am248@stud.uni-Heidelberg.de\\
  
\end{tabular}
}

{
  \textbf{GitHub Repository: \url{https://github.com/DonatJR/ita_ws20}}
}

\end{titlepage}

\pagenumbering{arabic} 
% TODO Add source link to Data repo, this is done in the text and not on frontpage?

\section{Motivation}

When starting a new project or trying to improve existing bodies of work it is often vitally important to do extensive research.
This both helps to familiarize oneself with the topic that is to be worked on and provides the necessary new information that is needed to solve the problem at hand.

In comparison to more general search engines, where finding information for most topics is rather easy and comfortable, doing research on scientific papers can be more cumbersome.
There are multiple websites with different papers and search interfaces available, and to get a comprehensive overview one has to go through all of them and adapt their `search-procedure' to the available tools.

We want to propose a solution to this by taking scientific texts and clustering them into relevant subgroups, which can then be more easily presented to and explored by people looking for specific topics and terms.
While we intend to first focus on a rather narrow subset of papers from a topic like Deep Learning or something similar, more or less specific subgroups for clustering therein can be explored to find a good balance.
The pipeline should later be usable on a broader range of fields.

Furthermore, we think a solution to the stated problem could later be used on a grander scale by building good group visualization tools and providing existing websites with this technology.
In addition, a meta search site incorporating data from these other sources with a common, easily digestible search and presentation interface could be developed.

\section{Research Topic Summary}
In order to get an overview of a research field without extensive research and reading countless papers, there are various approaches to simplify this. By browsing through digital libraries and search engines you can find papers by matching search strings, but to get an overview of an entire research field this is not enough.
In \cite{Rapid_understanding_of_scientific_paper_collections} a tool is presented that uses different methods to gain insights into research fields.
Through the citation network, papers can be divided into clusters. Furthermore, trends, gaps and outliers can be identified.
Another possibility to get insights is the citation context, because the key statements of the paper are often summarized concisely.  To know the key statements of the paper, it is still necessary to read the whole paper or at least all citation contexts. For many papers this is still a very large amount of work. To solve this problem Multi-Document Summarization is used here. This summarization is only applied to abstract and citation context. The dataset used is ACL Anthology Network (AAN).\cite{aan} The dataset contains the network of citations, as well as the full text of each article, its metadata, summary, references and citation sentences. 

With the help of the techniques mentioned, it should be possible to gain insight into an entire research field more quickly.
In this project we will concentrate on only one method, the clustering.
For many years, attempts have been made to cluster papers in order to simplify research. In 1973, for example, it had already been tried to cluster journals by comparing reference patterns and looking at mutual references \cite{Clustering_of_scientific_journals}.

In \cite{Document_clustering_of_scientific_texts_using_citation_contexts} the context of the citations is used in addition to the citations to cluster.
First a citation has to be recognized and the text has to be extracted on both sides of the citation. Then link-based clustering approaches, term-based clustering approaches and hierarchical document clustering, as well as a combination of all three, are applied and compared. In addition, this technique is also applied to the entire document and compared to the approach of citation context.

But there are also approaches where citations are not used.
In \cite{Clustering_scientific_documents_with_topic_modeling} abstracts and titles are used.
For this purpose the bibliography with various queries is downloaded from Web of Science\cite{web_of_science}.  
Two types of pre-processing are then performed on the texts. The first method treats each word as a token, and stopwords are deleted. The second method uses term-clumping to find noun phrases with significant commonality.
In addition, several topic modeling algorithms are used. 
The Latent Dirichlet allocation(LDA), Correlated Topic Models (CTM), Hierarchical Latent Dirichlet Allocation (Hierarchical LDA) and Hierarchical Dirichlet Process (HDP) are tested.
The clusters created by the algorithms have to be named manually.
Abstracts are also used in \cite{An_Approach_to_Clustering_Abstracts} to perform clustering.
First tokenization is used, then stopwords are removed and a stemming algorithm is performed.
Because of the shortness of abstracts, the words must have a higher frequency than in a general balanced corpus of the given language. Then the keywords are grouped and weighted and the closeness of two documents is calculated using cosine measure. 
Additionally, clustering methods are applied to the whole abstract. Three algorithms from three different approaches are used: the k-medoid method from the example-based approach, the nearest neighbor method from the hierarchy-based approach and the MajorClust method from the density-based approach.
As data source 48 abstracts from \cite{cicling} are used, which have been classified by a human.

\section{Project Description}

\subsection{Main project goals}

practicalities of your approach, this time with a clearer focus on the application:
- make it easier for users to search and explore scientific papers belonging to a specific topic / theme
- create cluster representation that makes this easy to include in downstream tasks

Our main goal is to make it easier for users to search and explore scientific papers belonging to a specific topic or theme.
For this we want to specifically arrive at a clustering (representation) that, for one, separates the different documents into correct clusters, but is also easy to work with in downstream tasks (e.g. the mentioned inclusion in some search site).
To achieve this we basically interpret the steps mentioned in the subsection \ref{subsec:pipeline} as some coarse sub goals, which can then be worked on by different team members.
Some of these sub goals can also be further divided, for example downloading and preprocessing data from different sources or implementing distinct clustering algorithms can be done by a single team member respectively.

\subsection{Pipeline}
\label{subsec:pipeline}

Our pipeline will consist of the stages that can be seen in figure \ref{fig:pipeline}. First of all data from two different sources will be downloaded, temporarily stored and cleaned up to disregard records with missing data.
All cleaned up records will be stored permanently so we don't have to download and clean up the data every time we change some downstream settings. From this data storage the text we are going to use in the end will be extracted and some clustering algorithms will be used to build the final result.


%\subsection{Data Set}
\label{subsec:data}
We have two main and one backup source of scientific papers:
\begin{itemize}
	\item Crossref \cite{crossref} is an official Digital Object Identifier Registration Agency that provides access to the full text of its registered research papers through their own API. Sending a HTTP request with "deep learning" as a query returns a JSON object with all the URLs to the PDFs of the registered papers found in Crossref's data base containing the terms "deep learning" in their title. The PDFs will then have to be converted into plain text.
	\item All About NLP \cite{aan} is a website maintained by Yale University's Learning and Information Group that provides a corpus consisting of over 400 scientific papers on NLP in plain text. Closer examination of the text files has, however, shown that most of them contain spelling mistakes.
	\item Journal of Machine Learning Research \cite{jmlr} is an international forum for the electronic and paper publication of scholarly articles in all areas of machine learning. The papers are available in pdf format, while the abstracts are in html format.
\end{itemize}


\newpage %TODO Delete me!
% TODO @Chen: Give an illustration maybe?
\section{Evaluation}
\label{sec:eval}
In this section, several evaluation approaches will be presented. We can distinguish between unsupervised and supervised clustering and will explore the implications for the evaluation. The overall processing pipeline was illustrated in the last section. Due to the inherent feedback loop between evaluation and processing, several consequences for the in detail processing pipeline will be drawn.

% TODO this will be moved
%\subsection{Goal}
%What is the goal of this work? Do we do unsupervised clustering of papers? Then we will have different evaluations on cluster statistics, number of clusters etc. 

% 1. We want to vanilla cluster text into segments and check whether this matches the host sites classification

% 2. We use a fixed set of label given by key word associations. We can classify each text and assign soft labels. Key word generation is then the process of e.g. taking the first k labels with highest classification score.

% 3. we generate hard assignments based on given key words. If we do not know hard clusters, but only key words. We can generate clusters based on key word similarities. The clustering algorithm does not know the key words, but can be supervised by the key word clustering. This introduces an additional algoritm in our pipeline.

\subsection{Unsupervised Clustering}
Unsupervised clustering denotes dividing data into clusters without supervision from a ground truth. Given a dataset $ \mathcal{D} $, our goal is to generate $ n $ separate clusters for each datum $ x \in \mathcal{D} $, where data within a cluster has \textit{maximal} similarity w.r.t. a specific data attribute. Clustering shares an inherent analogy to classification/assignment problem, as we are trying to find label, that assign each datum to a cluster/class. We can distinguish between \textit{hard} and \textit{soft} labels, depending on whether we would like to assign a single or multiple labels. Classification requires knowing the label set $ \mathcal{L} $, which may not be given for our project. We will first assume this to be true and perform unsupervised clustering. 

In order to separate elements $ x_{i},\; x_{j} \in \mathcal{D} $, a distance function is needed. 
\textbf{TODO}: Axioms distance function

%TODO (Chen): A large focus of this work will lie in the underlying vector embedding of the data. Clustering performance mainly depends on this
The choice for a distance function is as critical as the choice for an underlying metric space for the data. We believe that a large portion of our project will be spent with \textit{feature engineering}, where we compare clustering algorithms on different data representations. We will propose a learning based approach in the supervised scenario for a given clustering algorithm. To the best of our knowledge, this may not have been done for our project and is an interesting direction for future work.

Example algorithms for unsupervised clustering are:
\begin{itemize}
	\item K-Means
	\item Mean shift
	\item Spectral clustering, Graph cuts
	\item Agglomerative (/hierarchical) clustering
	\item Expectation-Maximization algorithm
	\item ...
\end{itemize}
We plan on trying out multiple algorithms for the text representations in the literature before going to the supervised scenario and creating our own representations. Efficient implementations of these algorithms exist in \textit{sklearn} and similar Python libraries. 

It is non-trivial to characterize clustering metrics. In general we can distinguish between \textit{internal} and \textit{external} measures. This directly corresponds to degrees of supervision. Because no ground truth exists, internal measures are highly dependent on the application. Typical internal measures depend on the number of clusters, intra- and inter-cluster distance distributions. Examples are the \textit{Davies-Bouldin index}, \textit{Dunn index} or the \textit{Silhouette coefficient}. However, we usually do not know much about the underlying data, which makes it hard to compare algorithms this way. This is also the case for this project. 

For example, \textit{Expectation maximization} uses an underlying distributional model to compare clusters, with which we can compute different distances than for a heuristic \textit{K-means} algorithm. But how would we know, that the underlying model assumptions are valid for our data type? Consequently, internal measures only indicate if one algorithms is better than another in some situations. In the end, we might however not know if an algorithms produces more valid results than another. If such a measure existed, we would have a priori knowledge that makes a comparison obsolete. Example: If we knew beforehand that our data has non-convex clusters, we would not need to use the K-means algorithm, which can only find convex clusters.

\textbf{TODO}: Add references to literature, etc.
\textbf{TODO}: Elaborate more on metrics with equations.

\subsection{Supervised Clustering}
Generating a supervision for clustering research papers is not a trivial task. There are several possible solutions for computing a supervisory signal. One of the general ideas of this project was to exploit the standardized submission process in academic journals/conferences. Oftentimes (/always?) authors need to submit a list of key words that best describe their work. On top, all documents follow structural guidelines for their specific journal/conference. This gives us the chance to use the standardized format to simplify processing. Furthermore, we can exploit the labels of platforms hosting our data. Articles are usually already grouped according to their topics. We propose two supervision modes that we want to try out: 
\begin{itemize}
	\item Create hard labels, by using the preexisting group assignments of hosting websites
	\item Create labels from provided key words upon submission. We could create a label set across the data and perform classification on our training set. This would allow to a) assign likely key words to papers b) cluster papers according to assigned key words. A down-stream task that benefits from our improvements would be key word generation. This would make the need for key word submission obsolete and researchers could focus on research.
	\item Generate hard assignments based on key words. By assigning cluster labels to key words, we could compute a hard cluster from a combination of key words for a given paper. This boils down to 1. and is sort of meta, as we cluster the supervision in order to get supervision for our actual task. Let $ \mathcal{K} $ denote the set of key words, $ \mathcal{N} $ the set of clusters and the function $ f\colon \mathcal{K} \mapsto \mathcal{N} $ our supervisory mapping function. If we invert $ f $, we can again generate key words similar to 2.
\end{itemize}

Supervised clustering allows validating an algorithm based on a training, test and validation split. Typical evaluation metrics include: \textit{Precision}, \textit{Recall}, \textit{F1-score}, \textit{Jaccard index}, \textit{confusion matrix}, etc.

With a supervision, we can further implement a learning algorithm. For example we could use the \textit{cross entropy} for our training loss (this is not directly evaluation, but closely related). We would like to find out how a neural network performs compared to traditional clustering algorithms on this task, as we did not found many papers that do so. Another possible direction in our project would be to use a learning algorithm to design the features, that we could cluster with a traditional algorithm and see whether we can find better features for our task. This point has not been explored much yet and is of utter importance.

%NOTE Some needs to be put into Research summary! 
% NOTE @Jessica: I can help you with your research summary and improve the section.

\newpage %TODO Delete me!

%\input{CH_outlook}

%%%%%%%%%%%%%%%%%%%%%%%%%%%%%%%%%%%%%%%%%%%%%%%%%%%%%%%%%%%%

% The following is especially useful if you work together on one proposal or report, and want to alter its content independently from each other (e.g., to keep your commit history clean).

% Alternative: put content in separate files
% Check the difference between including these files using \input{filename} and \include{filename} and see which one you like better
%\chapter{Einleitung}\label{intro}
%\section{Introduction}

\comment{Written by Daniela Fichiu}
If you type a query like "how do I open the terminal on mac?" into Google, you already know what you want to find: a set of steps that will open a terminal on the screen.

Most would say doing homework is, on the best days, an unpleasant affair. But everyone would agree that writing an assignment for that one course you've always skipped is a chore. You most certainly do not understand the airy slides. Your only option is to search for materials on your own - extensive research is needed to find the materials that can fill all your knowledge gaps.

A student who wants to know more about text analytics might search for "text analytics" on Google Scholar. The query yields back 1.650.000 results, the top three being "Text Analytics with Python," "Text analytics in social media," and "Semantic interaction for visual text analytics." A subsequent query like "text analytics topics" yields back results with the following titles: "Analyzing educational comments for topics and sentiments: A text analytics approach" or "A text analytics approach for online retailing service improvement: Evidence from Twitter."

We know how time-consuming it is to spend hours on search engines or websites like Research Gate or Google Scholar looking for research papers, hoping for the best, but never quite finding the perfect materials.

Our project emerged from the need to find an easy way of exhaustively searching for scholarly literature while bringing to light the relations between the subfields of the research field of interest and other areas. Our goal is to provide a deeper understanding of the material to be searched for and easing up the process of finding information.

We propose a solution that clusters scientific texts into relevant subgroups. The subgroups can then be more easily presented to and explored by people looking for specific topics and terms.

We focus on a subset of around two thousand papers from the field of machine learning. We cluster the research papers, extract a ground truth using the papers' keywords, and compare the clustering results against the ground truth. We also prove that an unbalanced data set can have a significant impact on the clustering results. 

Even with the conclusions written down, we do not see our work as finished. The proposed pipeline could be usable on a broader range of fields. We also intend to balance our data set by adding research papers from other areas.

We also believe that our solution, supported by good group visualization tools and a user-friendly search interface, can be successfully integrated into a meta-search site.
%
%\chapter{Voraussetzungen}\label{bg}
%\input{background}

%%%%%%%%%%%%%%%%%%%%%%%%%%%%%%%%%%%%%%%%%%%%%%%%%%%%%%%%%%%%

% References (Literaturverzeichnis):
% see
% https://de.wikibooks.org/wiki/LaTeX-W%C3%B6rterbuch:_bibliographystyle
% for the different formats and styles

\bibliographystyle{plain}
% b) The File:
\bibliography{bibtex/references, bibtex/CH_ref}

\newpage

\begin{appendices}
    \section{Figures}
    \begin{figure}[ht]
        \centering
        \includegraphics[width=\textwidth,keepaspectratio]{figures/pipeline}
        \caption[]{Coarse overview over the intended pipeline.}
        \label{fig:pipeline}
    \end{figure}
  \end{appendices}

\end{document}
